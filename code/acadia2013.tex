\documentclass[landscape,a0paper,fontscale=0.292]{baposter}

\usepackage[vlined]{algorithm2e}
\usepackage{times}
\usepackage{calc}
\usepackage{url}
\usepackage{graphicx}
\usepackage{amsmath}
\usepackage{amssymb}
\usepackage{relsize}
\usepackage{multirow}
\usepackage{booktabs}

\usepackage{graphicx}
\usepackage{multicol}
\usepackage[T1]{fontenc}
\usepackage{ae}
\usepackage[export]{adjustbox}
\usepackage{float}
\usepackage{epstopdf}
\usepackage{multicol}


\graphicspath{{images/}}

 %%%%%%%%%%%%%%%%%%%%%%%%%%%%%%%%%%%%%%%%%%%%%%%%%%%%%%%%%%%%%%%%%%%%%%%%%%%%%%%%
 %%%% Some math symbols used in the text
 %%%%%%%%%%%%%%%%%%%%%%%%%%%%%%%%%%%%%%%%%%%%%%%%%%%%%%%%%%%%%%%%%%%%%%%%%%%%%%%%
 % Format 
 \newcommand{\RotUP}[1]{\begin{sideways}#1\end{sideways}}


 %%%%%%%%%%%%%%%%%%%%%%%%%%%%%%%%%%%%%%%%%%%%%%%%%%%%%%%%%%%%%%%%%%%%%%%%%%%%%%%%
 % Multicol Settings
 %%%%%%%%%%%%%%%%%%%%%%%%%%%%%%%%%%%%%%%%%%%%%%%%%%%%%%%%%%%%%%%%%%%%%%%%%%%%%%%%
 \setlength{\columnsep}{0.7em}
 \setlength{\columnseprule}{0mm}


 %%%%%%%%%%%%%%%%%%%%%%%%%%%%%%%%%%%%%%%%%%%%%%%%%%%%%%%%%%%%%%%%%%%%%%%%%%%%%%%%
 % Save space in lists. Use this after the opening of the list
 %%%%%%%%%%%%%%%%%%%%%%%%%%%%%%%%%%%%%%%%%%%%%%%%%%%%%%%%%%%%%%%%%%%%%%%%%%%%%%%%
 \newcommand{\compresslist}{%
 \setlength{\itemsep}{1pt}%
 \setlength{\parskip}{0pt}%
 \setlength{\parsep}{0pt}%
 }

 %%%%%%%%%%%%%%%%%%%%%%%%%%%%%%%%%%%%%%%%%%%%%%%%%%%%%%%%%%%%%%%%%%%%%%%%%%%%%%%%
 %input defs file
 %%%%%%%%%%%%%%%%%%%%%%%%%%%%%%%%%%%%%%%%%%%%%%%%%%%%%%%%%%%%%%%%%%%%%%%%%%%%%%%%
 \input{defs}
 
%%%%%%%%%%%%%%%%%%%%%%%%%%%%%%%%%%%%%%%%%%%%%%%%%%%%%%%%%%%%%%%%%%%%%%%%%%%%%%%%
%input colors
%%%%%%%%%%%%%%%%%%%%%%%%%%%%%%%%%%%%%%%%%%%%%%%%%%%%%%%%%%%%%%%%%%%%%%%%%%%%%%%%
\definecolor{DarkBlue}{rgb}{0.1,0.1,0.5}
\definecolor{Red}{rgb}{0.9,0.0,0.1}

 

%%%%%%%%%%%%%%%%%%%%%%%%%%%%%%%%%%%%%%%%%%%%%%%%%%%%%%%%%%%%%%%%%%%%%%%%%%%%%
%% Begin of Document
%%%%%%%%%%%%%%%%%%%%%%%%%%%%%%%%%%%%%%%%%%%%%%%%%%%%%%%%%%%%%%%%%%%%%%%%%%%%%
\begin{document}
%%%%%%%%%%%%%%%%%%%%%%%%%%%%%%%%%%%%%%%%%%%%%%%%%%%%%%%%%%%%%%%%%%%%%%%%%%%%%
%% Here starts the poster
%%---------------------------------------------------------------------------
%% Format it to your taste with the options
%%%%%%%%%%%%%%%%%%%%%%%%%%%%%%%%%%%%%%%%%%%%%%%%%%%%%%%%%%%%%%%%%%%%%%%%%%%%%
\begin{poster}{
 % Show grid to help with alignment
 grid=false,
 % Column spacing
 colspacing=0.7em,
 % Color style
 headerColorOne=DarkBlue!30!Red,
 headerColorTwo=Red!30!DarkBlue,
 headerFontColor=white,
 borderColor=black,
 % Format of textbox
 textborder=roundedleft,
boxshade=plain,
boxColorOne=white,
boxColorTwo=cyan,
 % Format of text header
 headerborder=open,
 headershape=roundedright,
 headershade=plain,
 background=shadeTB,
 bgColorOne=cyan!10!white,
 bgColorTwo=gray,
 headerheight=0.12\textheight}
 % Eye Catcher
 {
      \includegraphics[width=0.08\linewidth]{atei}
      \includegraphics[width=0.08\linewidth]{CapeSplitHiking}

 }
 % Title
 {\sc\huge High Resolution Numerical Models of the Digby Neck Passages}
 % Authors
 {Mitchell O'Flaherty-Sproul, Richard Karsten, Joel Culina\\[1em]
 {\texttt{073208o@acadiau.ca, rkarsten@acadiau.ca, jculina@acadiau.ca}}}
 % University logo
 {
  \begin{tabular}{r}
    \includegraphics[height=0.11\textheight]{Acadia_Crest}
    %\raisebox{0em}[0em][0em]{\includegraphics[height=0.13\textheight]{Acadia_Crest}}
  \end{tabular}
 }

%%%%%%%%%%%%%%%%%%%%%%%%%%%%%%%%%%%%%%%%%%%%%%%%%%%%%%%%%%%%%%%%%%%%%%%%%%%%%%
%%% Now define the boxes that make up the poster
%%%---------------------------------------------------------------------------
%%% Each box has a name and can be placed absolutely or relatively.
%%% The only inconvenience is that you can only specify a relative position 
%%% towards an already declared box. So if you have a box attached to the 
%%% bottom, one to the top and a third one which should be inbetween, you 
%%% have to specify the top and bottom boxes before you specify the middle 
%%% box.
%%%%%%%%%%%%%%%%%%%%%%%%%%%%%%%%%%%%%%%%%%%%%%%%%%%%%%%%%%%%%%%%%%%%%%%%%%%%%%

%%%%%%%%%%%%%%%%%%%%%%%%%%%%%%%%%%%%%%%%%%%%%%%%%%%%%%%%%%%%%%%%%%%%%%%%%%%%%%
  \headerbox{Introduction}{name=intro,column=0,row=0,span=2}{
%%%%%%%%%%%%%%%%%%%%%%%%%%%%%%%%%%%%%%%%%%%%%%%%%%%%%%%%%%%%%%%%%%%%%%%%%%%%%%
\begin{small}Nova Scotians require electricity to be generated from renewable sources, and
tidal power can play a role in fulfilling this desire.
Understanding the energetic tidal currents in the Bay of Fundy is a necessary step in implementing tidal power systems.
Numerical models are a vital tool to increase our understanding.
Two grids were developed for use in the Finite Volume Coastal Ocean Model (FVCOM).
The grids, one with a resolution of 15 metres and the other with a resolution of 125 metres
were combined with high resolution bathymetry and coastlines to create accurate models.
This poster will present an analysis of the results for Grand Passage.
\end{small}

\vskip .3cm

\begin{center}
\begin{tabular}{ccc}


 \includegraphics[width=.3\textwidth,frame]{scotMaine_ll_gp_wireframe.jpg}
&
  \includegraphics[width=.3\textwidth,frame]{smallisland_gp_wireframe.jpg}
&
   \includegraphics[width=.3\textwidth,frame]{dg10m_24_10_ppolex_2_gp_wireframe.jpg}\\


 % lowres.jpg: 881x835 pixel, 72dpi, 31.08x29.46 cm, bb=0 0 881 835

\begin{normalsize}Original grid \end{normalsize}&
\begin{normalsize}125 Metre grid \end{normalsize}&
 \begin{normalsize}15 Metre grid \end{normalsize}
\end{tabular}

\end{center}


 }
 
%%%%%%%%%%%%%%%%%%%%%%%%%%%%%%%%%%%%%%%%%%%%%%%%%%%%%%%%%%%%%%%%%%%%%%%%%%%%%%
  \headerbox{ADCP Comparison}{name=adcpcomp,column=0,span=2,below=intro}{   
%%%%%%%%%%%%%%%%%%%%%%%%%%%%%%%%%%%%%%%%%%%%%%%%%%%%%%%%%%%%%%%%%%%%%%%%%%%%%%
\begin{small}Observations were collected as part of the Southwest Nova Scotia Tidal Energy Resource Assessment project. The ADCP data was processed by Justine McMillan.
Comparisons were performed between the numerical simulation results and observations. 
The plots below show remarkable agreement between the both models elevation when compared to ADCP observations.
The 125 metre model is over estimating the amplitude of the speed, the 15 metre model is not.
The 15 metre model is predicting fluctuations in speed which are essential to understanding the tidal currents in the passages.\end{small}



\begin{center}
\vskip -.3cm
\begin{tabular}{ccc}
    \includegraphics[clip=true,trim=5cm 0cm 10cm 0cm,width=.3\textwidth,keepaspectratio=true]{dg10m_gp_adcp_locations_2.jpg} &
       \includegraphics[width=.3\textwidth,keepaspectratio=true]{dg10m_gp_adcp4_to_grid_with_el_sub.eps}&
       \includegraphics[width=.3\textwidth,keepaspectratio=true]{smallisland_gp_adcp4_to_grid_with_el_sub.eps}\\
     ADCP locations  & 15 Metre Model & 125 Metre Model

\end{tabular}
\vskip .2cm
\begin{footnotesize}(Blue) ADCP data. (Red) Model data. Speed in m/s, elevation in metres.\\ The ADCP locations are plotted over water depth in metres.\end{footnotesize}
\end{center}
   

}



 %%%%%%%%%%%%%%%%%%%%%%%%%%%%%%%%%%%%%%%%%%%%%%%%%%%%%%%%%%%%%%%%%%%%%%%%%%%%%%
   \headerbox{Transects}{name=transect,column=2,row=0,span=2}{
 %%%%%%%%%%%%%%%%%%%%%%%%%%%%%%%%%%%%%%%%%%%%%%%%%%%%%%%%%%%%%%%%%%%%%%%%%%%%%%
 
\begin{small} The transect data was collected by mounting an ADCP on a boat and traveling across Grand Passage.
 The images below are cross sections of Grand Passage north of Peter's Island during flood tide.  
In the observations there is a large jet of high speed flow on right and a smaller jet on the left. The direction of the 
flow is northward, with the exception of the flow above the submerged tip of Peter's Island.
Here the flow curls back around and reverses direction, traveling to the south.
The 125 metre model does not predict the highlighted behaviours.
The 15 metre model does predict these behaviours; a high resolution model is necessary to simulate this level of detail.\end{small}


\begin{center}
\vskip -.35cm
 \begin{tabular}{ccc} 

    \includegraphics[width=.25\textwidth,keepaspectratio=true]{track_1_smallisland_gp_transect_speed_model.jpg}
    
  &
    \includegraphics[width=.25\textwidth,keepaspectratio=true]{track_1_dg10m_gp_transect_speed_model.jpg}
    
&
    \includegraphics[width=.25\textwidth,keepaspectratio=true]{track_1_dg10m_gp_transect_speed_obs.jpg}\\
  
    125 Metre Model Speed (m/s) & 15 Metre Model Speed (m/s) & Observed Speed (m/s)\\

    \includegraphics[width=.25\textwidth,keepaspectratio=true]{track_1_smallisland_gp_transect_direction_model.jpg}
&
    \includegraphics[width=.25\textwidth,keepaspectratio=true]{track_1_dg10m_gp_transect_direction_model.jpg}
&
    \includegraphics[width=.25\textwidth,keepaspectratio=true]{track_1_dg10m_gp_transect_direction_obs.jpg}\\
  
 125 Metre Model Direction ($\deg$) & 15 Metre Model Direction ($\deg$) & Observed Direction ($\deg$)
\end{tabular}\end{center}
 
 
 
  }

 %%%%%%%%%%%%%%%%%%%%%%%%%%%%%%%%%%%%%%%%%%%%%%%%%%%%%%%%%%%%%%%%%%%%%%%%%%%%%%
   \headerbox{Conclusions}{name=references,column=0,span=2,below=adcpcomp,above=bottom}{
 %%%%%%%%%%%%%%%%%%%%%%%%%%%%%%%%%%%%%%%%%%%%%%%%%%%%%%%%%%%%%%%%%%%%%%%%%%%%%%
 \begin{small}Two numerical models of the Digby Neck passages were developed for use in FVCOM.
 Both models were accurate, and able to simulate the tidal elevation.
 The 15 metre model predicted the vertical structure of the flow observed during the transects.
 The 15 metre model was also used to discern information about physical processes and explained the speed fluctuations which were in the ADCP observations.
 
 Similar analysis were performed on Petit Passage and Digby Gut and can be viewed in \textit{New High and Low Numerical Models of the Tidal Currents Through the Digby Neck Passages} by Mitchell O'Flaherty-Sproul.
 
 %High resolution numerical models provide a valuable tool for understanding the tidal currents in the Bay of Fundy.                                                                                                                
 \end{small}
 
   }



 %%%%%%%%%%%%%%%%%%%%%%%%%%%%%%%%%%%%%%%%%%%%%%%%%%%%%%%%%%%%%%%%%%%%%%%%%%%%%%
    \headerbox{Residual}{name=tracking,column=2,span=2,below=transect,above=bottom}{
 %%%%%%%%%%%%%%%%%%%%%%%%%%%%%%%%%%%%%%%%%%%%%%%%%%%%%%%%%%%%%%%%%%%%%%%%%%%%%%
\begin{small} The residual is tidally driven flow that has velocity fluctuations at different periods than the tidal constituents.
It is related to the interaction of the tides with the bathymetry and geographic features of a region.
Eddies form as the main tidal flow passes Peter's Island, the eddies are seen clearly in the residual.
The fluctuations in speed found in both the ADCP observations and numerical results happen as eddies pass over a location.
Using the numerical model to draw plots of the residual it was determined that the speed fluctuations in the observations are not stochastic turbulence.
The speed fluctuations are caused by coherent structures moving in the flow.

\end{small}
     
\begin{center}
\vskip -.35cm
    \begin{tabular}{cc}   
    \includegraphics[width=0.44\textwidth,keepaspectratio=true]{dg10m_gp_adcp5_to_grid_speed_with_el_sub.jpg}&  
    \includegraphics[clip=true,trim=50cm 12cm 60cm 70cm,width=0.38\textwidth,keepaspectratio=true]{{residual_step_1264}.jpg}   
    \end{tabular}
    
    
    
\begin{footnotesize}Speed in m/s, elevation in m.
(Blue) ADCP data. (Red) 15 Metre model data. 
Vertical black line in the left plot is the time which corresponds to the right plot. 
Left plot are timeseries at the location of the white marker in the right plot.\end{footnotesize}
    \end{center}
       

   
    }
    



\end{poster}%
%
\end{document}
