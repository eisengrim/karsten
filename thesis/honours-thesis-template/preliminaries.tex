%---------------------------------------------------------
% Preliminaries: Set up your own details in this file!
%----------------------------------------------------------

% Don't forget to remove the ()s in ALL of these "personalization" lines.

\title{(PUT YOUR THESIS TITLE IN preliminaries.tex)}
\author{(PUT YOUR FULL NAME IN preliminaries.tex)}
\dept{(PUT YOUR DEPT IN preliminaries.tex)}   % E.g., Physics, Computer Science,
\deptOrSchool{(Department or School)}         % Pick one, remove the rest
\degree{(PUT YOUR DEGREE IN preliminaries.tex)}  % E.g., Science, Arts, ...

\submissionMonth{(March)}	    % OR WHATEVER MONTH YOU ACTUALLY SUBMIT IN
\submissionYear{(20XX)}
\copyrightYear{(20XX)}		    % Probably the same as submissionYear.

% Use a "~" after the "r." of "Dr." so that TeX doesn't think you have
% ended a sentence (at which point it gives extra space).
\supervisor{(Dr.~Your Supervisor)}

% Remove the '%' from the next line and fill in the name if desired.
%\cosupervisor{(Dr.~Your Other Supervisor)}

\headOrDirector{(Dr.~The Head or Director)}
% If the head or director is an ``acting'' head or director, uncomment
% the next line (i.e., delete the '%'):
% \justActing

% You will have to ask around to find out the name of the person
% to put here... it changes from year to year.
\honoursCommittee{(The honours committee person)}

%-------------------------------------------------------------------------

% This outputs the title page, the approval page and the copyright page.
\firstThreePages

%-------------------------------------------------------------------------
% Now write your acknowledgements (if any).
% If you wish to acknowledge no-one, delete or comment-out the
% next few lines.

\Acknowledgments

Place any acknowledgments you might want to make here.

Don't forget to be formal and professional.

%-------------------------------------------------------------------------

% This outputs the table of contents, lists of figures, tables, ...

\tocAndSuch

%-------------------------------------------------------------------------

\prefacesection{Abstract}

This is a ``quick-start'' guide to help you use \LaTeX\ to produce
your thesis.  It by no means covers all issues, but it should give you
a solid start.

\TeX\ is a system developed by Donald Knuth with the goal of doing
beautiful typesetting, particularly for documents which use
mathematics.  It is one of the first significant programs made
available to the world at large \emph{for free} by its author.  By
design, Knuth has dictated that \TeX\ won't change any more, which
means that \TeX\ documents won't become unusable because of
incompatible updates.  (However, other people are extending \TeX, so
it is not a ``dead'' system.)

\LaTeX\ is a set of additions to \TeX\ which arguably make it easier
to produce documents.  This sample thesis will assume you are going to
use \LaTeX\, but if you later to decide to try Knuth's ``plain'' \TeX,
you will find that the concepts are fairly similar.

This document was initially written by Brian Demmings; any ``I'' found
in this document refers to Brian.  Jim Diamond has updated it
occasionally since Brian graduated, and hopes he has only improved
things with his changes.  Alex Sanford took the masters template and
transmogrified it into the honours thesis template.

The ``source code'' of this document should be in the same directory
as the nicely-formatted PDF file you are reading.  If you want to see
exactly how anything here is produced, just look in the appropriate
``source code'' file.  (For example\dots\ if you read the source code,
you will see that we used ``\verb|\dots|'' rather than ``\verb|...|'',
since the spacing of ``\dots'' is better than ``...''.)

\medskip

Write \emph{your} abstract here (in the file \verb|preliminaries.tex|).
		
\medskip

\textbf{BEWARE:} Some of the techniques (e.g., the ``Table of
Algorithms/Listings'') may \emph{not} be permitted.  When in doubt
check with RGS (masters students), the registrar (honours students) or
your advisor.


%-------------------------------------------------------------------------

% Don't mess with this line!
\afterpreface
