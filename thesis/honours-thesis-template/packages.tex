%---------------------------------------------------------
% Chapter: A chapter about packages
%----------------------------------------------------------
\chapter{Packages}

You might find some or all of the following packages useful for
writing your thesis.  To find out more options about these packages,
you can read the corresponding manual.  On a Linux system you should be
able to read (for example) the hyperref documentation by executing the
command \verb|texdoc hyperref|.


\begin{description}
    \item[textcomp] Useful for some extra symbols.
    \item[inputenc] Specifies the encoding format of inputs.
      Generally you would use \verb|latin1| for a thesis written in
      English.
    \item[amsmath] Many useful math mode bits. 
    \item[amsfonts] Fonts for use in math mode (e.g., subscript sized
    Greek letters). 
    \item[amssymb] Extra math symbols (not necessarily useful).
    \item[amsth] Allows you to define ``theorem-like'' sections
    (e.g., the definition used earlier).
    \item[graphicx] Package for graphics functionality and quality.
    \item[soul] Provides ability to ``\textbf{s}trike \textbf{o}ut''
      and \textbf{u}nder\textbf{l}ine text (e.g., useful at
    editing time).
    \item[hyperref] Provides hyper-linking within the PDF output file.
    % If
    % used with the \verb|linktocpage| option the page numbers in
    % the table of contents are hyperlinks to the corresponding pages.
      At time of writing, the options given to the hyperref package in
      this sample thesis' \verb|header.tex| cause the table of
      contents entries, figure and tables numbers, bibliographic
      references and various other things to be hyperlinks.  That is,
      if you see (for example) ``Figure~\ref{fig:EXAMPLE1}'' and you
      click on the ``\ref{fig:EXAMPLE1}'', your PDF viewer should take
      you to that figure.  (Try it out right now.)
      Note that in \verb|header.tex| there are a few fields you should
      edit by hand to register the author and title of your thesis
      inside the PDF file.
    \item[listings] A very powerful listings package to make life easier.
    \item[subfig] Allows you to create sub-figures and subtables (using
    \verb|\subfloat|) within figures and tables.  This replaces the old
    \verb|subfigure| package which is deprecated.
    \item[verbatim] Allows for the use of the \verb|verbatim| environment
    which makes it easier to include multi-line text verbatim in the
    document.
    \item[alltt] Allows for including text that is typed in typewriter
    font. The text is interpreted as code, so be careful of the \TeX\ and 
    \LaTeX\ commands you use.
    \item[natbib] Better control over your bibliography, including
    citation style within the document (e.g., saying \citet{1188581}
    versus just \citep{1188581}).  Numerous customization parameters. 
    \item[acadia-hon-thesis] The ``hopefully-done (for 2015/16,
      anyway)'' Acadia honours style package which was used for this
      sample thesis.
    \item[algorithm] Adds powerful/clean algorithm functionality. 
    \item[algorithmic] Adds an algorithmic environment to your
    documents.  Allows for labeling, captioning, etc.
    \item[tikz] This is a very powerful package for producing many
    different kinds of plots, diagrams, graphs, and so on.  Including
    this package gives you the basic capability; to get additional
    capabilities you need additional \verb|\usepackage| commands (put
    them in \verb|header.tex|).  For example, the data plots shown in
    Chapter~\ref{chap:GRAPHICS} require the use of
    \verb|\usepackage{pgfplots}|.
\end{description}
