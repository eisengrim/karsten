
%-----------------------------------------------------------------------------
% Chapter: 
%-----------------------------------------------------------------------------

\chapter{Acronyms and Lists}
\label{chap:ACRONYMS_AND_LISTS}

\section{Defining an Acronym Command}

\TeX\ and \LaTeX\ have very powerful facilities for creating your own
commands.  So powerful, in fact, that discussing them is well out of
the scope of this document.  However, we will present one simple
example here which you may find useful.

Suppose you have some long word or phrase which is used frequently in
your thesis (such as ``deoxyribonucleic acid''), but you don't want to
type it over and over.  If you put the following command in your
document header (\verb|header.tex| in this sample thesis)
\begin{verbatim}
\def\dna{deoxyribonucleic acid}
\end{verbatim}
then you can just type \verb|\dna| and \LaTeX\ will expand that into
what you really wanted.  For example, if you now have this text

\def\dna{deoxyribonucleic acid}

\begin{verbatim}
I try to eat some \dna\ every day.
I like \dna.
\end{verbatim}
you will get this output:

\vspace{6 pt}
I try to eat some \dna\ every day.
I like \dna.

\vspace{6 pt}

% Hi, if you are looking at the source code, we note that the \def
% just has to appear before the first use, not necessarily in the
% header.

There are \LaTeX\ add-on packages to do all sorts of things for you,
and one such package handles acronyms in a much more flexible way.
For example, you might want to define an acronym so that the first
time it is used you get something like ``Turing Machine (TM)'' and
from then on the acronym expands to just ``TM''.  If this possibly
interests you, try doing a search at the Comprehensive TeX Archive
Network (CTAN) (\url{http://www.ctan.org}).

One final note about this\dots\ as mentioned elsewhere,
\LaTeX\ ``eats'' spaces following commands such as ``\verb|\dna|''.
In order to get a space, the first ``\verb|\dna|'' needs to be
followed by something which causes the space to be preserved.  In the
second ``\verb|\dna|'', it is immediately followed by punctuation
(``\verb|.|'' in this case) and so there is no need to do anything
special.  If you have a hard time remembering this and don't mind a
bit of extra typing, you could always write ``\verb|{}|'' after
commands, as follows:

\begin{verbatim}
I try to eat some \dna{} every day.
I like \dna{}.
\end{verbatim}
you will get this output:\\
I try to eat some \dna{} every day.
I like \dna{}.




%------------------------------------------------------------------------------
% Section: Lists
%------------------------------------------------------------------------------

\section{Examples of Lists}
\label{sec:LISTS}
There are numerous examples of places that we might want to use a
list.  We might use a described list, which is begun with
\verb|\begin{description}|, ended with \verb|\end{description}|, and
where each item is introduced with \verb|\item[...text...]|.  The
``\verb|...text...|'' for the first item below is ``\verb|First Item|''.
\begin{description}
    \item[First Item] This thesis talks about \sysacro.
    \item[Second] This thesis talks about XML \citep{1188581}.
    \item[Final Point]  This thesis talks about the future!
\end{description}

\noindent Or we might want to use a bulleted list, started with
\verb|\begin{itemize}|, and each \verb|\item| then has no
\verb|[...text...]|.

\begin{itemize}
    \item This thesis talks about \sysacro.
    \item This thesis talks about XML \citep{1188581} by \citealt{1188581}.
    \item This thesis talks about the future!
\end{itemize}

\noindent Or we might use a numbered list, where each item is
introduced with \verb|\begin{enumerate}|, and each \verb|\item| again
has no \verb|[...text...]|.
\begin{enumerate}
    \item This thesis talks about \sysacro.
    \item This thesis talks about XML \citep{1188581} by \citealt{1188581}.
    \item This thesis talks about the future!
\end{enumerate}

\noindent The point is that we have a great deal of flexibility.

One thing that you might not like about the above lists is that there
are lots of blank lines.  If we add
\verb|\vspace{-\topsep}|
\emph{before} the \verb|\begin{itemize}| line and
\emph{after} the \verb|\end{itemize}| line and
\vspace{-\topsep}
\begin{description}
    \setlength{\itemsep}{0pt}
    \setlength{\parskip}{0pt}
    \setlength{\parsep}{0pt}
    \item[\quad] \verb|\setlength{\itemsep}{0pt}|
    \item[\quad] \verb|\setlength{\parskip}{0pt}|
    \item[\quad] \verb|\setlength{\parsep}{0pt}|
\end{description}
\vspace{-\topsep}
\emph{after} the \verb|\begin{itemize}| line, the above bulleted list
will look like this:
\vspace{-\topsep}
\begin{itemize}
    \setlength{\parskip}{0pt}
    \setlength{\itemsep}{0pt}
    \setlength{\parskip}{0pt}
    \setlength{\parsep}{0pt}
    \item This thesis talks about \sysacro.
    \item This thesis talks about XML \citep{1188581} by \citealt{1188581}.
    \item This thesis talks about the future!
\end{itemize}
\vspace{-\topsep}
that is, with no extra space before, after, or between list items.

Budding \LaTeX\ gurus may prefer to read about and then use the
\verb|enumitem| package which makes this a lot easier to do.
